\documentclass[english]{article}
\usepackage{/home/paul/toolchain/R/R-1.7.0/share/texmf/Sweave}
\begin{document}

%\VignetteIndexEntry{Cookbook for Monitoring Models Guide}


\section{Cookbook for Monitoring Models}
In R, the functions in this package are made available with

\begin{Schunk}
\begin{Sinput}
> library("monitor")
\end{Sinput}
\end{Schunk}

This section gives a brief recipe for building short term forecasting
models. It is intended to be self-contained although there are references
to other sections for additional information.

The term "monitoring" comes from the fact that one is often
trying to monitor the current state of the economy based on data from
prior periods, since there is typically some lag before statistical
agencies release data for the current period. The steps, explained
in more detail below, are:


1/ specify the data series to use in the model


2/ estimate a model and confirm that it is reasonable


3/ repeat 1 and 2 if other series are to be considered for competing
models (beware that fishing can be dangerous)


4/ run the monitoring program to produce forecasts


and optionally


5/ set up an automatic program to run the monitoring program and
distribute results


This library use the TS PADI interface explained in more detail in
an appendix. For example purposes it is assumed that the data can
be retrieved from an "economic time series" (ets) server. The
examples use names of series which are used internally at the Bank
of Canada and are probably not available elsewhere. Start S/R and
open a graphics window with

\begin{Schunk}
\begin{Sinput}
> x11()
\end{Sinput}
\end{Schunk}

If running remotely it may be necessary to use an argument 
like "-display YourWorkstation:0.0" to display on your workstation. A
few more details on running S/R are given in Section 2 of this guide.


\subsection{Step 1- specify the data}


The data is specified in an variable which indicates the name of
the series, the source, any transformations which should be applied,
and possibly some other options. For more details see the section on
\emph{TSdata} in the guide for dse1.
An example of a model which contains two outputs and no inputs is

\begin{Schunk}
\begin{Sinput}
> if (require("dsepadi")) cbps.manuf.data2.ids <- TSPADIdata2(output = list(c("ets", 
      "", "i37013", "percent.change", "cbps.prod."), c("ets", "", 
      "i37005", "percent.change", "manuf.prod.")), pad.start = FALSE, 
      pad.end = TRUE)
\end{Sinput}
\end{Schunk}

With the above, the data will be converted to percent change when
it is read from the database. The default behaviour for data retrieval
is to trim all series to the same length. The length is such that
there are no missing values on the ends. pad.start and pad.end can
be used to modify this behaviour. With pad.end=TRUE all series are padded
on the end with NAs to give a length which will include the most recent
data value from any series. This is preferred for forecasting but
the NAs have to be trimmed with trim.na for estimation procedures.
The data is actually retrieved from the database with

\begin{Schunk}
\begin{Sinput}
> if (require("padi") && checkPADIserver("ets")) cbps.manuf.data2 <- freeze(cbps.manuf.data2.ids)
\end{Sinput}
\end{Schunk}

This example and others below will not work without a database server
that provides the indicated data. The \emph{if} in the above allows automatic
example checking to work (at the Bank of Canada).

The following example specifies one input series and one output series.
It uses an alternate constructor (TSPADIdata vs. TSPADIdata2) which
takes arguments in a different format. (The result is the same but
different styles sometimes seem more convenient.)

\begin{Schunk}
\begin{Sinput}
> manuf.data.ids <- TSPADIdata(input = "lfsa455", input.transforms = "percent.change", 
      input.names = "manuf.emp.", output = "I37005", output.transforms = "percent.change", 
      output.names = "manuf.prod.", server = "ets", pad.start = FALSE, 
      pad.end = TRUE)
> if (require("padi") && checkPADIserver("ets")) manuf.data <- freeze(manuf.data.ids)
\end{Sinput}
\end{Schunk}

The data can be plotted with

\begin{Schunk}
\begin{Sinput}
> if (require("padi") && checkPADIserver("ets")) tfplot(manuf.data)
\end{Sinput}
\end{Schunk}

In this example the plot shows missing data in the middle. In this
somewhat unusual case it is necessary to trim the beginning of the
data set to remove the portion up to the end of the missing data.
This could be done with

\begin{Schunk}
\begin{Sinput}
> if (require("padi") && checkPADIserver("ets")) manuf.data <- tfwindow(manuf.data, 
      start = c(1976, 2))
\end{Sinput}
\end{Schunk}

However, the trimming would have to be repeated each time the data
is updated from the database, which is especially inconvenient for
automatic procedures described further below. A better way is to set
the starting period for retrieved data with

\begin{Schunk}
\begin{Sinput}
> manuf.data.ids <- modify.TSPADIdata(manuf.data.ids, start = c(1976, 
      2))
\end{Sinput}
\end{Schunk}

then when data is retrieved with

\begin{Schunk}
\begin{Sinput}
> if (require("padi") && checkPADIserver("ets")) manuf.data <- freeze(manuf.data.ids)
\end{Sinput}
\end{Schunk}

it will start after the missing data. The start can also be specified
with the argument start for the function TSPADIdata.


A more detailed plot of the last portion of the data can be produced
with

\begin{Schunk}
\begin{Sinput}
> if (require("padi") && checkPADIserver("ets")) tfplot(manuf.data, 
      start. = c(1995, 11))
\end{Sinput}
\end{Schunk}

Note the "." after start is part of the name of the argument.
It is often not necessary since truncated arguments usually match
without problem, but is required in the case of tfplot so that the
argument is not confused with the function start. To specify and retrieve
data with two input series and one output series

\begin{Schunk}
\begin{Sinput}
> cbps.manuf.data.ids <- TSPADIdata(input = c("lfsa462", "lfsa455"), 
      input.transforms = "percent.change", input.names = c("cbps.emp.", 
          "manuf.emp"), output = "i37013", output.transforms = "percent.change", 
      output.names = "cbps.prod.", start = c(1976, 2), server = "ets", 
      db = "", pad.start = FALSE, pad.end = TRUE)
> if (require("padi") && checkPADIserver("ets")) cbps.manuf.data <- freeze(cbps.manuf.data.ids)
\end{Sinput}
\end{Schunk}

To specify and retrieve data with one input variable and two output
variable

